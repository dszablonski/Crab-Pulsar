% !BIB TS-program = biber
\documentclass{beamer}
\usetheme{CambridgeUS}
\setbeamertemplate{caption}[numbered]
\usepackage{amsmath}
\usepackage{subcaption}

\usepackage{physics}
\usepackage[backend=biber, style=nature]{biblatex}
\usepackage{siunitx}
\usepackage{animate}

\newcommand{\comment}[1]{}

\title{Crab Pulsar}
\subtitle{}
\author{Dominik Szablonski and Nicholas Smith}
\date{\today}

\begin{document}
	\begin{frame}
		\titlepage
	\end{frame}
	\section{Introduction}
	\begin{frame}{What is a Pulsar?}
		\begin{figure}
			\centering
		\animategraphics[loop,width=0.9\textwidth]{10}{pulsar_}{1}{48}
		\caption{https://lilith.fisica.ufmg.br/~dsoares/extn/ogs/ogs-psr.htm}
		\end{figure}
	\end{frame}
	\begin{frame}{Aims and Goals of the Experiment}
		\begin{enumerate}
			\item Measure DMs of multiple pulsars and find their distance from the earth.
			\item Measure the period and identify pulsars by Fourier analysis of their data.
			\item Measure the period of the Crab Pulsar a period of time using pulsar timing techniques to find its period derivative, and hence estimate its surface magnetic field and age.
		\end{enumerate}
	\end{frame}
\end{document}